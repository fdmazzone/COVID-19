\documentclass{article}

\usepackage{fancyhdr} % controla los headers and footers.
\usepackage{color}
\usepackage{setspace}
\usepackage{amsmath,amsthm}
\usepackage{epsfig}
\usepackage{color, tcolorbox}
 
\usepackage[utf8]{inputenc}
\usepackage[spanish, activeacute]{babel} 
\usepackage{epsfig}
\usepackage{amssymb}
\usepackage{latexsym}
\usepackage{amssymb}
\usepackage[figuresright]{rotating}
\usepackage{pdflscape}

\usepackage{amsfonts}
\usepackage{graphicx,epsf}
\usepackage{natbib}
\usepackage[font=scriptsize,labelfont=bf]{caption}
\usepackage{graphicx,multicol}
\usepackage{booktabs}  
\usepackage{float}
\usepackage[section]{placeins}
\usepackage[caption = false]{subfig}
\usepackage{multirow}
%\usepackage[toc,page]{appendix}
\usepackage{pdflscape}
\usepackage{multirow}
\usepackage{url}
\usepackage{marvosym}

\topmargin 0in
\oddsidemargin -0.05in
\evensidemargin -0.05in
\textwidth 6 in
\textheight 8.5in
\parskip 0.1in




%%%%%%%%%%%%%%%%%%%%%%%%%%%%%%%%%%%%%% definiciones

\def\dst{\displaystyle}


\def \espe  {\text{E}} 
\def \var  {\text{VAR}} 
\def \ecm  {\mbox{\sc ecm}}
\def \var  {\mbox{\sc var}}
\def \ses  {\mbox{\sc ses}}
 \def\median{\mathop{\rm mediana}}
\def\sin{\mbox{sen}}
\def\real{\hbox{$\displaystyle I\hskip -3pt R$}}
\def\natu{\hbox{$\displaystyle I\hskip -3pt N$}}
\def\partender{\hbox{$\displaystyle |\hskip -2.5pt ]$}}
\def\partenizq{\hbox{$\displaystyle |\hskip -4pt [$}} 
\def\colgar{\hangindent=2.3truecm \hangafter=1}


%%%%%%% cierre de una demostración con el cuadrado alineado
\def\qed {{% set up
		\parfillskip=0pt % so \par doesnt push \square to left
		\widowpenalty=10000 % so we dont break the page before \square
		\displaywidowpenalty=10000 % ditto
		\finalhyphendemerits=0 % TeXbook exercise 14.32
		%
		% horizontal
		\leavevmode % \nobreak means lines not pages
		\unskip % remove previous space or glue
		\nobreak % don?t break lines
		\hfil % ragged right if we spill over
		\penalty50 % discouragement to do so
		\hskip.2em % ensure some space
		\null % anchor following \hfill
		\hfill % push \square to right
		$\square$% % the end-of-proof mark
		%
		% vertical
		\par}} % build paragraph
 %%%%%%% 
 
 
\newcommand\e {\text{e}}
\newcommand\pc {\mbox{\tiny\sc PC}}
\newcommand\h {\widehat{h}_n}
\newcommand\ta {\widehat{\tau}_n}
\newcommand\ho {h_{ {\mbox{\tiny opt}},n}}
\newcommand{\cs}{  \stackrel{c.s.}\longrightarrow  }
\newcommand{\po}{  \stackrel{p}\longrightarrow  }
\def\square{\ifmmode\sqr\else{$\sqr$}\fi}
\def\sqr{\vcenter{
         \hrule height.1mm
         \hbox{\vrule width.1mm height2.2mm\kern2.18mm
\vrule width.1mm}
         \hrule height.1mm}}
\newcommand{\ed}{  \stackrel{d}=  }				
\newcommand{\sumi}{ \sum_{i=1}^n}					

%\setlength{\oddsidemargin}{22pt} \setlength{\textwidth}{6.0in}
%\setlength{\topmargin}{0.1in} \setlength{\textheight}{9.0in}
%\setlength{\headheight}{0in} \setlength{\headsep}{0.5in}
%\setlength{\topsep}{0in} \setlength{\itemsep}{0in}
\newtheorem{ejer}{Ejercicio}
 \newtheorem{teorema}{Teorema} 
\newtheorem{proposicion}{Proposición} 
\newtheorem{lema}{Lema} 
\newtheorem{observacion}{Observación} 
\newtheorem{corolario}{Corolario} 
\newtheorem{conje}{Conjetura} 
\newtheorem{hipotesis}{Hipótesis}
\newtheorem{definicion}{Definición} 
\newtheorem{ejemplo}{Ejemplo} 
\newtheorem{problema}{Problema} 

%%%%%%%%%%%%%%%%%%%%%%%%%%%%%%%%%%%%%% Estilo de pagina

\pagestyle{fancy}
\fancyhead{} % clear all fields
\addtolength{\headwidth}{\marginparsep} %esta sentencia y la proxima controlan
%la linea que divide a los headers and footers del resto del texto
\addtolength{\headwidth}{\marginparwidth}
 \lhead{}
 \chead{}\fancyhead{} % clear all fields
 \lhead{\bfseries \tiny{    Departamento de Matemática\\
 		Facultad de Ciencias Exactas, Físico-Químicas y Naturales\\
 		Universidad Nacional de Río Cuarto \\Abril de 2020.}}


	
	
	
	


\renewcommand{\headrulewidth}{0.2pt}
\renewcommand{\footrulewidth}{0.4pt}
\renewcommand{\baselinestretch}{1.0}

\setlength{\oddsidemargin}{22pt} \setlength{\textwidth}{6.0in}
\setlength{\topmargin}{0.1in} \setlength{\textheight}{8.1in}
\setlength{\headheight}{0.3in} \setlength{\headsep}{0.5in}
\setlength{\topsep}{0in} \setlength{\itemsep}{0in}



 \flushbottom


\begin{document}

\date{}

\begin{center}
 {\bf \large{Seminario de formación e investigación \\ ``Tópicos de la modelización y estimación de la propagación de Covid-19''
 		\\
 }}
\vspace{0.5cm}

\end{center}





\section{Contexto institucional}



 \subsection{Introducción}

La Organización Mundial de la Salud reportó un nuevo coronavirus el que fue identificado por las autoridades de la República de China el 7 de enero de 2020 como el causante de casos de neumonía en la ciudad de Wuhan (ver \cite{chen2020mathematical}). Este agente etiológico fue identificado como un nuevo betacoronavirus - perteneciente a la misma familia que el  SARS-CoV y el MERS-CoV-  mediante la aplicación de técnicas de secuenciación de nueva generación (NGS) de virus cultivados o directamente de muestras recibidas de diversos pacientes con neumonía, tal como reporta el Informe de la OMS \cite{OMS1}. A dicho agente se le denominó   Covid-19 (ó 2019-nCoV).




La OMS, en la persona de su director general Tedros Adhanom Ghebreyesus, declaró el 11 de marzo de 2020 que  Covid-19 es una pandemia.


Al día   30/04/2020  a nivel mundial hay  3.282.240 de casos confirmados de coronavirus,   232.180  personas muertas y 1.034.725  sujetos recuperados según el informe diario publicado en  ``Worldometers''\footnote{https://www.worldometers.info/coronavirus/} . En Argentina según el Reporte Diario vespertino del   29/04/2020 del Ministerio de Salud de la Nación hay  4285 casos confirmados o positivos con un total de 214 personas fallecidas. 

Los problemas críticos que surgen con la pandemia deben ser abordados desde diferentes campos científicos, principalmente la epidemiología, la virología y  la ciencia de la salud pública tal como se señala en la sesión especial del día del 08/03/2020  de la Conferencia CROI   (ver \cite{croi}). 

Experiencias anteriores muestran la importancia de la modelización matemática al momento de dar recomendaciones acerca de las medidas que se pueden adoptar frente a las epidemias, tal como se fundamenta  en el informe  ``Alerta y Respuesta Mundiales''  \citep{OMS} entre varios informes de la OMS. Como señala Hyndman en \cite{hyn1} los problemas implicados en  la modelización son complejos especialmente el de la predicción. 

  





Diferentes grupos de investigadores aportan desde la modelización matemática,  computacional y estadística desde diferentes universidades e institutos en el mundo, en particular en las UUNN  e institutos de investigación del  Consejo Nacional de Investigaciones Científicas  y Técnicas (CONICET)   en Argentina. En nuestro país se conformó un grupo multidisciplinar integrado por alrededor de 60 investigadores y becarias/rios del Instituto de Cálculo y  el Instituto de Ciencias de la Computación  como así también de otras unidades académicas de la la Facultad de Ciencias Exactas y Naturales de la Universidad de Buenos Aires y CONICET, y  de fuera de ella \cite{coldm}. Este ámbito tiene  interacción con la  Secretaría de Articulación Científica  del Ministerio de Ciencia, Tecnología e Innovación (MINCYT)  tal como se detalla en \cite{icuba}. También el grupo integrado por la Dra. Raquel Crescimbeni, la Dra. Alejandra Perin y el Dr. Luis Novak de la Universidad Nacional del Comahue han efectuado  aportes desde la matemática con fines educativos tales como el material audiovisual ``El teorema del Umbral: razones para declarar la cuarentena total''.  A nivel mundial hay diversos grupos de científicos involucrados en la modelización y, sólo a modo de ejemplo, mencionamos al grupo  `` COVID-19 Response Team'' del Imperial College London \citep{fergu2020} en Inglaterra, a  PRISM \citep{prism} en Australia y al grupo de Princeton University \cite{princet} en Estados Unidos. 
 


 Sistemas complejos, optimización, dinámica no lineal y estocástica, grafos, mecánica estadística, modelos lineales generalizados son algunas de las líneas que participan del problema de la modelización.
% En Argentina diferentes grupos se encuentran actualmente implementando modelos matemático-computacionales y de ciencias de datos aplicados a la pandemia del coronavirus, como por ejemplo, el  coordinado por el Dr. Guillermo Durán del el Instituto de Cálculo dependiente del Consejo Nacional de Investigaciones Científicas  y Técnicas (CONICET) y de la Facultad de Ciencias Exactas y Naturales de la Universidad de Buenos Aires (UBA) o de Hernán Solari del Departamento de Física de la UBA. 


%Otros grupos con extensa trayectoria en este tipo de modelizaciones es el de ``Modelado y simulación de la transmisión de enfermedades infecciosas'' coordinado por el Dr. Gabriel Fabricius del Instituto de Investigaciones Fisicoquímicas Teóricas y Aplicadas (INIFTA) dependiente de la Universidad Nacional de La Plata (UNLP) y CONICET. 

%

%(El grupo integrado por el Hugo Aimar y etc?).


Actualmente se están implementando políticas desde los estados nacionales y desde organismos multilaterales, como la OMS, para realizar aportes a la solución de la grave crisis que la pandemia está generando en el contexto de la humanidad toda. La crisis tiene una primera dimensión  prioritaria, la salud de la población y, también dimensiones complementarias y no por ello menos importantes como la económica y cultural.

 Ejemplos de aportes desde dispositivos específicos del ámbito científico-académico  se han dado en el marco de las convocatorias provinciales y nacionales  como las ``Jornadas COVIDLab Córdoba'' efectuada por el Ministerio de Ciencia y Tecnología de Córdoba como así también la convocatoria Ideas Proyecto (IP) ``COVID 19''  de la Agencia Nacional de Promoción de la Investigación, el Desarrollo Tecnológico y la Innovación. Esta última contó con 710 presentaciones. 
 
 %https://www.argentina.gob.ar/noticias/unidad-coronavirus-se-presentaron-710-propuestas-de-investigacion-desarrollo-e-innovacion 
 
 En nuestro país no podemos dejar de mencionar el importante esfuerzo que educadores, administrativos y estudiantes están realizando para sostener el proceso de enseñanza--aprendizaje en todos los niveles del sistema educativo nacional.



\subsection{Objetivos}

La propuesta de creación de este seminario cuenta con los siguientes objetivos institucionales.

\begin{itemize}
\item   {\bf Objetivo general} Aportar soluciones, desde el campo científico-educativo, a la crisis provocada por la pandemia del Covid-19.


\item {\bf Objetivos específicos}  (en relación a la pandemia del coronavirus).
\begin{itemize}
	\item Generar un ámbito institucional local (UNRC) de estudio en torno al problema modelización   y estimación de la propagación de Covid-19.
	
	\item Propiciar la interacción de investigadores y estudiantes locales con los grupos de investigación que ya están abordando el problema de la modelización de esta pandemia en las UUNN e institutos de investigación del país. 
	
	\item Promover el intercambio interdisciplinario en relación a una problemática compleja.
	
	\item Realizar desde los Programas y Proyectos de Investigación de la SeCyT-UNRC aportes disciplinares específicos (ecuaciones diferenciales, control óptimo, aprendizaje- learning- estadístico, etc). 
	
	\item Colaborar con la sistematización y tratamiento estadístico de datos que describen diferentes realidades de la dinámica de la enfermedad.  
	
	\item Aportar, a través de prensa institucional de FCEFQyN y de UNRC, a la comunidad de Río Cuarto y zona una mayor comprensión de la dinámica de la pandemia a los fines de fortalecer la salud de la ciudadanía.	
	
	
\end{itemize}

\end{itemize}



 A continuación en la Sección \ref{proble}  presentamos algunos de los problemas de la matemática  involucrados en la modelización.  En la Sección \ref{metocrono} presentamos la medodologia de trabajo y contenidos mínimos. En la Sección \ref{integrantes} presentamos la nómina de integrantes del seminario. Al final damos las refencias bibliográficas.

 

 



\section{Problemas matemáticos y computacionales de la modelización y del análisis de datos} \label{proble} 




\subsection{Reseña histórica} \label{rhist} 

\subsubsection*{Enfoque histórico general.}
A lo largo de la historia se han reportado diferentes plagas que azotaron a la humanidad. La primera epidemia significativa descripta por los historiadores fue la plaga de Atenas (430--426 a..C.). En 165-180 a.C., el Imperio Romano y Egipto fueron afectados por la viruela. Decenas de millones de personas murieron. Una de las epidemias mejor documentadas que devastó Europa fue la Peste Negra, la cual se extendió por todo el Mediterráneo y Europa y se estima que mató a unos 50--100 millones de personas en los años 1348--1350. Otra epidemia desastrosa atacó a la población azteca en el siglo XVI. Esta epidemia de viruela mató a unos 35 millones de personas. A principios del siglo XX, una pandemia de gripe mató a unos 20 millones de la población mundial. En la actualidad, todavía tenemos brotes significativos de epidemias: la peste de Bombay 1905-1906, el síndrome respiratorio agudo severo de 2003 (SARS) y la pandemia de gripe porcina H1N1 de 2009. Amenazas de epidemias existen continuamente, ya que los virus mutan muy rápidamente y pueden saltar barreras de especies, que infectan a los humanos, potencialmente a gran escala.

\subsubsection{Enfoque histórico específico.}

Aunque la epidemiología en sí tiene una larga historia, el estudio matemático de las enfermedades y su propagación solo tiene unos 350 años. El primer estudio estadístico de enfermedades infecciosas se atribuyen a John Graunt (1620--1674), cuyo libro de 1663 ``Natural
and Political Observations Made upon the Bills of Mortality'' se refería a métodos de estadísticas de salud pública. Un siglo después, Daniel Bernoulli utilizó métodos matemáticos para analizar la mortalidad por viruela. En 1766, publicó lo que ahora se considera el primer modelo epidemiológico (revisado en \cite{bernoulli2004attempt}). Una reformulación contemporánea del enfoque de Bernoulli en términos de ecuaciones diferenciales se da en \cite{dietz2002daniel}.

 A mediados del siglo XIX, Louis Pasteur hizo avances notables en las causas y la prevención de enfermedades tales como la rabia y el ántrax. Casi al mismo tiempo, el fundador de la bacteriología moderna, Robert Koch, identificó los agentes causantes específicos de la tuberculosis, cólera y ántrax, dando así apoyo experimental al concepto de enfermedad infecciosa. También fue famoso por el desarrollo de los postulados de Koch. A fines del siglo XIX, la ciencia finalmente pudo explicar el mecanismo de cómo uno se enferma. 

El concepto de transmitir una enfermedad bacteriana a través del contacto entre un individuo infectado y uno sano se hizo conocido. Esto allanó el camino para el modelado matemático de enfermedades infecciosas. Dicho modelado avanzó significativamente con la obra de William Hamer, a principios del siglo XX. Parece que Hamer fue el primero en usar la ley de acción de masas en el modelado de enfermedades infecciosas. Pero es Sir Ronald Ross quien es considerado el padre de la epidemiología matemática moderna. En su trabajo pionero sobre la malaria  descubrió que ésta se transmite entre humanos y mosquitos, estudió la dinámica del contagio y diseño estrategias para controlar la enfermedad. Ross recibió el Premio Nobel en 1902 por estos descubrimientos. En la segunda edición de su libro ``La prevención de la malaria'', publicado en 1911, desarrolló modelos matemáticos de la transmisión de la malaria y dedujo una cantidad límite, hoy en día conocida como el número de reproducción básica. 
La epidemiología matemática fue elevada a un nuevo nivel por el modelo de difusión de enfermedades infecciosas, publicado por Kermack y McKendrick en 1927. En su artículo ``Una contribución a la teoría matemática de las epidemias'' \cite{kermack1927contribution},  se publicó por primera vez un modelo epidémico determinista que incluía individuos susceptibles, infectados y removidos,  abreviadamente SIR \citep{dietz}.   Los mencionados autores publicaron luego la Parte II y la Parte III de esta obra en 1932 y 1933 respectivamente. La modelización matemática de las enfermedades infecciosas ganó importancia en los años ochenta con el advenimiento de las epidemias de VIH. Desde entonces, una gran cantidad de modelos han sido creados, analizados y empleados para estudiar la propagación de enfermedades infecciosas. Hoy, la epidemiología matemática tiene una presencia constante en la literatura de investigación, y el modelado matemático está haciendo contribuciones significativas a la salud pública \cite{hethcote1994thousand,vynnycky2010introduction}.

 

\subsection{Clasificación de los modelos} \label{clasif}

Los modelos matemáticos en general  pueden clasificarse utilizando diversos criterios. A los modelos epidemiológicos podemos clasificarlos basados en un criterio fenomenológico que atienda a los procesos dinámicos que queremos reflejar en ellos. Esta dinámica puede ser compleja y con características disímiles para diferentes enfermedades lo que determina modelos matemáticos también diversos. Entre algunas de las características dinámicas que se  estudian en la literatura (ver \cite{FredBrauer}) citamos

\begin{enumerate}
 \item Medios de contagio, esto es a través de contacto con otros individuos infectados de la misma población o a través de vectores (mosquitos, vinchuca, etc).
 %\item   Las vías de transmisión sexual.
 \item Factores demográficos, nacimientos y muertes de individuos, migraciones.
 \item Evolución de la infecciosidad en los individuos enfermos.  A modo de ejemplo, si son plenamente contagiosos desde el momento de contagiarse o transcurre un tiempo hasta que se convierten en individuos contagiosos.
 \item Producción de inmunidad a la enfermedad.
 \item Efectividad  de las medidas de control de la enfermedad, vacunaciones, cuarentenas. Objetivo, optimizar estos medios de control.
 \item Las distinciones de los individuos susceptibles de enfermarse, según sexo, edad, condición social, hábitos.
 \item Los requrimientos sanitarios como el de la hospitalización. 
 \item La distribución espacial de la infección.
 
 \end{enumerate}
 
 Cada una de estas características dinámicas lleva a distintas formulaciones en los  modelos, incluso pude hacer que ellas requieran para su formulación objetos matemáticos de distinta naturaleza  acorde a esas características, por ejemplo un modelo que contemple la distribución geográfica de la población suele involucrar ecuaciones en derivadas parciales a diferencia de los modelos que desestiman aquella distribución, los cuales suelen formularse con ecuaciones diferenciales ordinarias. Se sabe que la etapa de la epidemia en que se esté transitando puede hacer conveniente utilizar modelos estocásticos en lugar de deterministas o viceversa.
 
 
 Los modelos también pueden clasificarse acorde a las estructuras y objetos matemáticos que intervienen en su formulación. Podemos en primera instancia clasificarlos en dos grandes áreas: deterministas y estocásticos. Los modelos deterministas usualmente involucran ecuaciones diferenciales ordinarias, ecuaciones en derivadas parciales, ecuaciones de recurrencia. Los modelos estocásticos involucran cadenas de Markov, u otro tipo de procesos estocásticos, ecuaciones diferenciales estocásticas, laplacianos fraccionarios entre otras herramientas.  
 
 
 \subsection{Abordaje metodológico de la modelización } \label{metmodel}
 
 
 Casi todos los modelos suponen una división o \emph{compartimentación} de la población en clases. Sólo a modo de ejemplo en los  modelos de  Kermack y McKendrick la población fue dividida en tres clases: susceptibles  $S$, infectados $I$ y removidos $R$.  El modelo es construido a partir de suposiciones, leyes, que dan cuenta de la dinámica de como individuos de una clase son incorporados a otra. En estas leyes es común que  intervengan parámetros que no son conocidos a priori, su determinación es un problema. Las leyes que se formulan suelen plasmarse en ecuaciones en recurrencia, diferenciales ordinarias, diferenciales parciales o estocásticas. Pueden aparecer también ecuaciones con delay o laplacianos fraccionarios. Esquemáticamente tenemos:
 \begin{itemize}
 	\item Parámetros que representamos por un vector $\mu\in\mathbb{R}^n$
 	\item  Funciones, digamos a modo de ejemplo $S(t,\mu)$, $I(t,\mu)$ y $R(t,\mu)$. Estas funciones surgen de la solución del problema matemático formulado en el modelo, las ecuaciones que supone el modelo por ejemplo.
 \end{itemize}
 
 Luego de formular un modelo matemático suele ser necesario ``sintonizar'' el mismo con los datos que ofrece la realidad. Encontrar parámetros que hagan que el modelo refleje aceptablemente bien la evolución observada es un criterio para aceptar la \emph{validez} del modelo. Este proceso de validación puede ser llevado a cabo tomando en consideración los datos históricos de la epidemia. Esto supone contar con observaciones históricas de las variables, digamos $S_i$, $T_i$ y $R_i$, efectuadas en tiempo pasados $t_i$, $i=1,\ldots,n$. Luego es posible comparar las observaciones con las predicciones del modelo y tratar de estimar los parámetros que produzcan el error mínimo, por ejemplo el error cuadrático
 \[
 E=\sum_{i=1}^n|S_i-S(t_i,\mu)|^2+\sum_{i=1}^n|I_i-I(t_i,\mu)|^2+\sum_{i=1}^n|R_i-R(t_i,\mu)|^2.
 \]
 En este proceso se puede tomar en cuenta el error estadístico de  las observaciones y analizar su propagación al resultado hallado a modo de calcular  rangos de certidumbres en la estimación de los parámetros. Sólo para ilustrar, para el modelo SIR - de  Kermack y McKendrick-  existe un  umbral   del número básico de reproducción $\mathcal{R}_0$   y para el cual hay diferentes formas de estimarlo dependiendo del tipo de datos disponibles y estudiar, por ejemplo, el efecto del tamaño muestral en la estimación a través de intervalos de confianza \citep{KR}.
 
  
 Lo anterior apunta a establecer herramientas  que nos permitan predecir la evolución cuantitativa de la epidemia. También es pertinente y útil el estudio cualitativo del modelo.  Por ejemplo, puede ser probado por medios análíticos que para el modelo SIR de 
 Kermack y McKendrick, existe un  umbral   del número básico de reproducción $\mathcal{R}_0$ (concretamente $\mathcal{R}_0=1$) tal que si $\mathcal{R}_0<1$ la epidemia se extingue y para $\mathcal{R}_0>1$ la epidemia evoluciona hasta un pico máximo para luego decrecer. 
  
 
 \subsection{Aportes específicos desde la teoría del control óptimo } 

 
 La modelización matemática de enfermedades infecciosas ha demostrado que las combinaciones de aislamiento, cuarentena, vacuna y tratamiento son a menudo necesarias para eliminar la mayoría de estas enfermedades. Sin embargo, si no se administran en el momento adecuado y en la cantidad correcta, la eliminación de la enfermedad seguirá siendo una tarea difícil. La teoría del control óptimo ha demostrado ser una herramienta exitosa para comprender formas de reducir la propagación de enfermedades infecciosas e idear estrategias óptimas de intervención en enfermedades. La idea principal detrás del uso del control óptimo en epidemias es buscar, entre las estrategias disponibles, la más efectiva que permita reducir la tasa de infección al mínimo nivel mientras se optimiza el costo de implementar una terapia o vacuna preventiva que se utiliza para controlar la progresión de la enfermedad \cite{GerardoChowell487}. 
 
  
 
 En términos de enfermedades epidémicas, tales estrategias pueden incluir terapias, vacunas, aislamiento y campañas educativas \cite{BKO,C}. Recientemente, han aparecido muchos modelos de control óptimo de enfermedades epidémicas en la literatura. Incluyen, pero no se limitan a, modelo de epidemia de SIRS retrasado \cite{AAC}, modelo SIR retrasado \cite{ALA,BGO}, modelo de tuberculosis \cite{ST}, modelo de VIH \cite{HY} y dengue \cite{AGS}.
 \newline 
 
 
 
 
 \subsection{Aportes desde la estadística-matemática y la ciencia de datos} 
  
La estadística matemática en el contexto de la denominada  ciencia de los datos juegan roles auxiliares y  complementarios en este tipo de modelización y un rol central en la organización, visualización y comunicación de los datos en torno a la pandemia. 


Como se mencionó arriba, la estadística participa del problema de la estimación de los parámetros de los modelos descritpos, como los del tipo SIR.   

Al mismo tiempo es posible establecer buenas predicciones de corto plazo utilizando el análisis de series de tiempo basado en el suavizamiento exponencial como en los recientes trabajos de \cite{JZC} y \cite{PM}. Un estudio introductorio, detallado y actual de este tipo de pronóstico puede hallarse en el texto  \cite{AthaHynd}.  Un abordaje exahustivo al suavizamiento exponencial puede encontrarse en \cite{Hyndvarios}. 

En \cite{coldm} se menciona  la importancia de la estadística en
\begin{itemize}
	\item la utilización de regresión logística para determinar cómo afectan los diferentes factores de riesgo a la probabilidad de que un enfermo de Covid19 necesite cuidados intensivos,
	\item la construcción de un método de aprendizaje automático que, utilizando los datos observados, pueda predecir si un nuevo paciente necesitará o no cuidados intensivos. 
	\item  las predicciones sobre la evolución de curvas de contagio utilizando regresión no lineal, de Poisson, métodos basados en componentes principales funcionales y componentes principales funcionales dinámicos.  
\end{itemize}





Un tercer rol consiste, teniendo en cuenta los avances en las técnicas de   visualización, en organizar y comunicar la enorme masa de datos vinculada a la pandemia con fines educativos.  A modo de ejemplo ver la discusión en torno a la utilización de los log ratios dada por Hyndman en \citep{hyn2}.


 
 \subsection{Consideraciones metodológicas generales} 
  
 Para finalizar la presente Sección 2 introducimos brevemente perspectivas metodológicas y epistemológicas globales.
 
 
  Como señalan Keeling y Rohani \cite{KR}   es importante tener en cuenta que  en torno a  la modelización  existe una tensión, en general no resoluble, entre ``precisión''(habilidad para reproducir los datos observados y una dinámica futura confiable),  ``transparencia'' (comprensión de cómo los componentes del modelo influyen en la dinámica y cómo interactúan)  y  ``flexibilidad'' (facilidad de adaptación del modelo a nuevas situaciones, lo que es vital para evaluar políticas de control o para predecir niveles futuros de la enfermedad en un contexto cambiante).
 
 
 Como indica \cite{hyn1} hay que ser cautos en la utilización de modelos de series temporales ya que los mismos son buenos cuando los datos son precisos, el futuro es similar al pasado cuando no hay otros modelos del proceso subyacente. Precisamente este no es el escenario de la pandemia de COVID-19. Hyndman sugiere la utilización de modelos más generales como aquellos  basados en agentes. Una publicación reciente que utiliza modelos de compartimentación  incorporando complejidad de escenarios es la realizada en \cite{fergu2020}. 
 
 La epidemia ha provocado una crisis social, económica y cultural cuya complejidad sólo puede ser abordada desde diferentes disciplinas. Esta propuesta de seminario es un recorte a la problemática.
 
 
	
	


\section{Metodología de trabajo, cronograma y  contenidos mínimos} \label{metocrono}

La metodología de trabajo se basará en encuentros virtuales y con un cronograma ajustado al nuevo calendario institucional aprobado por la FCEFQyNat de la UNRC.



Se abordarán los siguientes temas. 

En el marco de este seminario:
\begin{itemize}
	\item[a)]   En una primera etapa se llevará  a cabo un estudio de modelos compartimentados. Se introducirán los modelos SIR de Kermack y McKendrick. Se estudiarán resultados cualitativos en torno a su dinámica. Se abordarán modelos más complejos que incluyen otros compartimentos, por ejemplo para individuos infectados pero no infecciosos, indidicuos aislados, en cuarentena y se discutirá el efecto de la dinámica demográfica en la evolución de la epidemia.  Se elaborarán programas que ajusten los datos de la epidemia del COVID-19 a los modelos estudiados. Se estimarán parámetros de la epidemia, por ejemplo el número básico de reproducción $\mathcal{R}_0$. Se trazaran líneas de acción a seguir, se propondrán problemas y delinearan estrategias de abordaje, por ejemplo conformación de subgrupos que aborden distintas problemáticas, como ser estrategias de control, estudio específico de la evolución de la epidemis en nuestra región, estudio de escenarios, estudios de otras epidemias (dengue, zika, chikungunya), entre otras cuestiones. La bibliografía que se propone es \cite{FredBrauer,MaiaMartcheva480,FredBrauer478,FredBrauer479} y algunos de los artículos que tratan específicamente con la modelación de la epidemia del COVID-19 mencionados en la bibliografía.
	
	\item[b)] 
	En relación a las estrategias de control y  siguiendo a \cite{MaiaMartcheva480}, abordaremos el tratamiento de  algunos modelos matemáticos específicos tales como: modelado de vacunación en enfermedades de una sola cepa, vacunación y diversidad genética de microorganismos, modelado de cuarentena y aislamiento. Luego,  se presentará la teoría básica del control óptimo a los efectos de estudiar teoremas de existencia y unicidad, condiciones de optimalidad, ecuaciones adjuntas, etc. \cite{SS,LAH,LW}. La misma proporcionará un fundamento matemático riguroso en el tratamiento de estos temas. Posteriormente, se considerarán algunos ejemplos sobre los cuales se mostrarán aplicaciones específicas de la teoría de control desarrollada. Finalmente, se realizará una revisión de trabajos científicos tales como \cite{ALA,AAC,AGS,BKO,BGO,C,HY,ST,ZKJ} a los efectos de formular y estudiar problemas de control óptimo vinculados particularmente a la pandemia de Covid-19.
	\item[c)] Desde la estadística abordaremos el problema de la estimación de los parámetros de los modelos de compartimentación como así también estudiaremos modelos de tipo   no lineal y suavizamiento exponencial utilizando la bibliografía  
	\cite{AthaHynd, Hyndvarios, KR, JZC, PM,  hyn1, hyn2, fergu2020}. 	\end{itemize} 
 


\section{Integrantes y coordinaciones}\label{integrantes}

Las siguientes personas - docentes, investigadores, asesores de PPI, becarios y adscriptos-  del Departamento de Matemática de la Facultad de Ciencias Exactas Físico-Químicas y Naturales de la Universidad Nacional de Río Cuarto integran la siguiente nómina inicial la cual permanece abierta a futuras incorporaciones.

\begin{itemize}
	\item  Barberis, Patricia
	\item  Beltritti, Gastón
	\item  Bollo, Carolina
	\item  Brunetto, Gisela
	\item  Buri, Leopolda
	\item  Cassano, Valentín
	\item  Colonna, Juan
	\item  Demaria, Stefanía
	\item  Ferreyra, David
	\item  Gariboldi, Claudia
	\item  Giubergia, Graciela
	\item  Maero, Andrea
	\item  Matos, Noelia
	\item  Mazzone, Fernando
	\item Navarro, Victoria
	\item Ruiz, Marcelo
	\item  Sorribes, Aylen
	\item Terráneo, Gabriel
	\end{itemize}



\section{Bibliografía}

En esta sección exponemos un sumario de una exploración  de bibliografía atinente a modelos epidemiológicos.

\subsection{Modelos epidemiológicos en general}

A continuación enumeramos libros y un artículo que abordan el problema de modelos epidemiológicos en general. Clasificamos las temáticas en cuatro dimensiones
\begin{description}
 \item[Estructura Matemática] Teorías matemáticas que intervienen en la formulación del modelo. 
 \item[Característica Epidemias] Que aspectos fenomenológicos son considerados en los modelos, diferenciación por edad, vacunaciones,  por ejemplo.
 \item[Estudios de caso] brotes concretos estudiados y analizados. 
 \item[Problemas considerados] Que problemas matemáticos son considerados, modelización, validación, estimación de parámetros, implementaciones numéricas. 
\end{description}

La siguiente tabla refleja que temas son tratados en la bibliografía explorada.


\begin{tabular}{|l|l|l|l|l|l|l|l|l|} \hline
                                           &                          & \cite{FredBrauer478}  & \cite{MaiaMartcheva480}    & \cite{FredBrauer}    &\cite{AllenSto}& \cite{AllenPri}&  \cite{calafiore}  & \cite{boto}                       \\\hline
\multirow{7}{*}{Estructura-Matemática}     & EDO                      &  \Checkedbox          &  \Checkedbox               &  \Checkedbox         & \Checkedbox   & \Checkedbox    &  \Checkedbox       &                                   \\\cline{2-9}
                                           & Teoría de Control        &                       &  \Checkedbox               &                      &               &                &                    &                                   \\\cline{2-9}
                                           & Procesos estocásticos    &  \Checkedbox          &                            &  \Checkedbox         &  \Checkedbox  & \Checkedbox    &                    &                                   \\\cline{2-9}
                                           & PDE                      &  \Checkedbox          &  \Checkedbox               &                      &               &                &                    &                                   \\\cline{2-9}
                                           & Ecuaciones Recurrencia   &                       &   \Checkedbox              &  \Checkedbox         &               &                &                    &                                   \\\cline{2-9}
                                           & Grafos                   &  \Checkedbox          &                            &  \Checkedbox         &               &                &                    &                                   \\\cline{2-9}
                                           & Laplaciano Fraccionario  &                       &                            &                      &               &                &                    & \Checkedbox                       \\\hline
\multirow{7}{*}{Característica Epidemias}  & Modelos Compartimentados & \Checkedbox           &  \Checkedbox               &  \Checkedbox         & \Checkedbox   & \Checkedbox    &  \Checkedbox       &                                   \\\cline{2-9}
                                           & Modelos Estocásticos     & \Checkedbox           &                            &  \Checkedbox         & \Checkedbox   & \Checkedbox    &                    &                                   \\\cline{2-9}
                                           & Modelos con Redes        & \Checkedbox           &                            &  \Checkedbox         &               &                &                    &                                   \\\cline{2-9}
                                           & Trasmisión vectores      &                       &   \Checkedbox              &  \Checkedbox         &               &                &                    &                                   \\\cline{2-9} 
                                           & Cepas múltiples          &                       &   \Checkedbox              &                      &               &                &                    &                                   \\\cline{2-9}
                                           & Distribución espacial    & \Checkedbox           &   \Checkedbox              &  \Checkedbox         &               &                &                    &                                   \\\cline{2-9}   
                                           & Estructurados por edad   & \Checkedbox           &   \Checkedbox              &  \Checkedbox         &               &                &                    &                                   \\\cline{2-9}
					   & Vacunaciones-cuarentenas & \Checkedbox           &   \Checkedbox              &  \Checkedbox         &               &                &                    &                                   \\\hline
\multirow{11}{*}{Estudios de caso}         & Dengue                   &                       &                            &  \Checkedbox         &               &                &                    &                                   \\\cline{2-9}
					   & Zika                     &                       &                            &  \Checkedbox         &               &                &                    &                                   \\\cline{2-9}
					   & Chikungunya              &                       &                            &                      &               &                &                    &                                   \\\cline{2-9}
                                           & Ébola                    &          	      &                            &  \Checkedbox         &               &                &                    &                                   \\\cline{2-9}
                                           & Gripe Aviar (H5N1)       &  		      &  \Checkedbox               &                      &               &                &                    &                                   \\\cline{2-9}
                                           & Gripe Porcina (H1N1)     &  	              &                            &                      &               &                &                    &                                   \\\cline{2-9}
                                           & COVID-19 (SARS-CoV2)     &  	              &                            &                      &               &                &  \Checkedbox       &                                   \\\cline{2-9}
                                           & HIV                      & \Checkedbox           &                            &  \Checkedbox         &               &                &                    &                                   \\\cline{2-9}
                                           & Tubeculosis              &                       &                            &  \Checkedbox         &               &                &                    &                                   \\\cline{2-9}
					   & Malaria                  &                       &                            &  \Checkedbox         &               &                &                    &                                   \\\cline{2-9}
                                           & Influenza                & \Checkedbox           &                            &  \Checkedbox         &  \Checkedbox  & \Checkedbox    &                    &                                   \\\hline
\multirow{4}{*}{Problemas considerados  }  & Modelización             & \Checkedbox           &  \Checkedbox               &  \Checkedbox         &  \Checkedbox  & \Checkedbox    & \Checkedbox        & \Checkedbox                       \\\cline{2-9}
                                           & Estimación Paramétrica   &                       &  \Checkedbox               &  \Checkedbox         &               &                & \Checkedbox        &                                   \\\cline{2-9}
                                           & Estudio cualitativo      & \Checkedbox           &  \Checkedbox               &  \Checkedbox         &  \Checkedbox  &  \Checkedbox   &                    &                                   \\\cline{2-9}
                                           & Simulación numérica      &                       &  \Checkedbox               &                      &               &  \Checkedbox   &                    &                                   \\\hline
\end{tabular}


\subsection{Modelado de la pandemia COVID-19}

Enumeramos artículos muy recientes (gran parte de ellos aún son preprints y no han pasado por un proceso de referato) tratando específicamente con la pandemia de la COVID-19. Clasificamos estos artículos acorde al tipo de modelo que desarrollan.
\begin{description}
 \item[Modelos con derivada fraccionarias] \cite{shaikh2020mathematical}.
 
\item[Modelos compartimentados] SEIR \cite{shi2020seir}, SIDARTHE \cite{giordano2020sidarthe}, SEIRNDC \cite{lin2020conceptual}, SEIRU \cite{liu2020model}, SEIQRDP \cite{peng2020epidemic}, SEIHR \cite{choi2020estimating, ivorra2020mathematical, delphi}, SIRU \cite{liu2020predicting}. 

\item[Modelos con tiempo discreto] SIRD tiempo discreto \cite{calafiore, anastassopoulou2020data}.

\item[Modelos estocásticos]  \cite{KUCHARSKI2020, HELLEWELL2020e488,fergu2020}.

\item[Modelos y técnicas de Control] \cite{djidjou2020optimal}.

\end{description}

 \bibliographystyle{plain}
 
\bibliography{Epidemias}



\end{document}


